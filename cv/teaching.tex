%-------------------------------------------------------------------------------
%	SECTION TITLE
%-------------------------------------------------------------------------------
\cvsection{Teaching and Workshops}

\cvsubsection{University Courses (Teaching Assistant)}

%-------------------------------------------------------------------------------
%	CONTENT
%-------------------------------------------------------------------------------
\begin{cventries}

%---------------------------------------------------------
  \cventry
    {University of Florida} % Organization
    {Introduction to Applied Statistics} % Training Title 
    {} % Location
    {Spring/Fall 2018} % Date(s)
    {
      \begin{cvitems} % Description(s) of tasks/responsibilities
    \item {Online Course (~120 to 150 students)}
        \item {Description: Conceptual and practical understanding of the application of statistics in the agricultural and life sciences using a combination of lectures, programming demonstrations, data exercises using the programming language R, group activities, and primary literature to teach introductory statistics at the graduate level}
        \item {Duties: Grading assignments, monitoring and responding to weekly discussion boards, serving as first contact for questions on content and issues with quizzes/exams, proofing assignments and final exam, holding weekly office hours}
        \item {Created \href{https://sta6093.justinmillar.com/}{additional notes and content} and \href{https://justinmillar.github.io/STA6093/ps01_nba-scout.html}{a comprehensive bonus assignment}}
      \end{cvitems}
    }

  \cventry
    {University of Florida} % Organization
    {Introduction to Bayesian Statistics} % Training Title 
    {} % Location
    {Spring 2016} % Date(s)
    {
      \begin{cvitems} % Description(s) of tasks/responsibilities
        \item {Description: Introduce life scientists to Bayesian statistics. We will explore basic ideas regarding integration through simulation (Monte Carlo integration), the philosophy and strengths of Bayesian statistics, and the Markov Chain Monte Carlo (MCMC) algorithms needed to fit such models. We will focus on several real world examples and how to transform these problems into statistical models. This course will rely on substantial extra-class work, in order to provide students with extensive hands on experience on conceptualizing, implementing, and interpreting the results of these models. Ideally, this experience will be enough to enable students to develop their own Bayesian models after this course is over. We will try to cover simple (e.g., Normal and Poisson), mixed effect and multi-level regression models but this will fundamentally depend on the speed with which the class is able to follow the course. Implementation of these models will be done both with JAGS as well as customized R code}
        \item {Duties: Grade weekly assignments (focused on R and JAGS code), grade group presentations, organize weekly "queries" from students, hold weekly office hours}
        \item {Motivated groups to develop webpages outlining their projects and contribute to \href{http://bayescourse.updog.co/}{new course website}}
      \end{cvitems}
    }

  \cventry
    {University of Mississippi} % Organization
    {Advanced General Microbiology} % Training Title 
    {} % Location
    {Spring 2012} % Date(s)
    {
      \begin{cvitems} % Description(s) of tasks/responsibilities
        \item {Description: A survey of the principles and concepts of microbiology including the biochemistry, cell biology, metabolism, genetics, ecology, evolution, and biodiversity of microorganisms, as well as the impacts of microorganisms on human affairs}
        \item {Duties: Prepare and maintain material for lab (mix and pour agar plates, culture strains), write and grade quizzes, hold open lab hours}
      \end{cvitems}
    }

  \cventry
    {Cape Eleuthera Institute} % Organization
    {Mangrove Flats Field Ecology} % Training Title 
    {} % Location
    {Summer 2011} % Date(s)
    {
      \begin{cvitems} % Description(s) of tasks/responsibilities
        \item {Small group (6) of high school juniors/seniors}
        \item {Description: Provide hands-on field research in near-shore mangrove ecology.}
        \item {Duties: Supervise weekly field sampling (free dive fish identification, seine netting, fish tagging), write and grade quizzes and assignments, prepare students for presentation for Bahamian Minister of the Environment}
      \end{cvitems}
    }

  \cventry
    {Michigan State University} % Organization
    {Applied Research Techniques (a.k.a. the Popcorn Course)} % Training Title 
    {} % Location
    {Spring 2011} % Date(s)
    {
      \begin{cvitems} % Description(s) of tasks/responsibilities
        \item {Description: Introduce students to how food companies do research to improve their products, and to have students carry out a research project of their own. With generous financial support from ConAgra Foods (makers of Peter Pan Peanut Butter, Reddi-Wip, and Parkay Margarine, among others), students will conduct experiments on a genuine research question associated with either ConAgra's Orville Redenbacher or Act II microwave popcorn}
        \item {Duties: Supervise student-developed research projects, create and grade quizzes and group project assignment, travel with class to ConAgra head-quarters and support student presentations to Reseach and Development team}
      \end{cvitems}
    }

  \cventry
    {Michigan State University} % Organization
    {Cell and Molecular Biology Lab} % Training Title 
    {} % Location
    {Fall 2010} % Date(s)
    {
      \begin{cvitems} % Description(s) of tasks/responsibilities
        \item {Description: Study of the building blocks of cells, the gross anatomy of the cell, and the structures and organelles that perform the work necessary for cell function. We will also examine several cellular processes at the molecular level, including the central dogma of molecular biology: RNA transcription and protein translation. We will examine the bioenergetic processes necessary to sustain life; first photosynthesis, the mechanism by which plant chloroplasts capture light energy to make the carbohydrates that bring life to earth. We then discuss the mitochondria and how they break down carbohydrates to release energy. All topics will be framed within the context of the human physiology. Mastery of these topics will provide you with an understanding of modern molecular and cellular biology}
        \item {Duties: Prepare lab materials, supervise student labs, grade lab reports, hold open lab hours}
      \end{cvitems}
    }

\end{cventries}


%-------------------------------------------------------------------------------
%	SUBSECTION TITLE
%-------------------------------------------------------------------------------
\cvsubsection{Workshops \& Tutorials}


%-------------------------------------------------------------------------------
%	CONTENT
%-------------------------------------------------------------------------------
\begin{cventries}

\cventry
    {Software \& Data Carpentry} % Organization
    {} % Training Title 
    {} % Location
    {University of Florida} % Date(s)
    {
      \begin{cvitems} % Description(s) of tasks/responsibilities
        \item {Served as an instructor and/or a helper in multiple (at least ten) Software (SC) and  Data (DC) Carpentry workshops (SW: The Unix Shell, Version Control with Git, Programming with Python, Programming with R, R for Reproducible Scientific Analysis; DC: Data Organization in Spreadsheets for Ecologists, Data Cleaning with OpenRefine for Ecologists, Data Management with SQL for Ecologists, Data Analysis and Visualization in R for Ecologist)}
        \item {Piloted an \href{https://datacarpentry.org/blog/2018/04/dc-seven-weeks}{experimental extended workshop series}}
        \item {Organized one of the first Data Carpentry Geospatial workshops}
      \end{cvitems}
    }

\cventry
    {R-Ladies} % Organization
    {} % Training Title 
    {} % Location
    {Gainesville, FL} % Date(s)
    {
      \begin{cvitems} % Description(s) of tasks/responsibilities
        \item {\href{https://www.justinmillar.com/r-ladies-blogdown.html}{Introduction to blogdown}}
      \end{cvitems}
    }

\cventry
    {UF R Meetup} % Organization
    {} % Training Title 
    {} % Location
    {University of Florida} % Date(s)
    {
      \begin{cvitems} % Description(s) of tasks/responsibilities
        \item {\href{https://justinmillar.github.io/leaflet-intro/}{Introduction to Leaflet in R}}
        \item {\href{http://www.r-gators.com/2018/03/28/introduction-to-shiny/}{Introduction to Shiny}}
        \item {\href{http://www.r-gators.com/2017/09/06/introduction-to-dataframes-in-r/}{Introduction to Dataframes in R}}
        \item {\href{http://www.r-gators.com/2017/08/30/introduction-to-r-and-rstudio/}{Introduction to R and RStudio}}
      \end{cvitems}
    }


\cventry
    {Undergraduate Statistics Club - R Tutorials} % Organization
    {} % Training Title 
    {} % Location
    {University of Florida} % Date(s)
    {
      \begin{cvitems} % Description(s) of tasks/responsibilities
        \item {\href{https://docs.google.com/presentation/d/1wLYdzfTozv4Ua9NcGBkFsLCsZIIpxa_4aHGOmYWh-p0/edit?usp=sharing}{Introduction to ggplot}}
        \item {Introduction to R and RStudio}
      \end{cvitems}
    }

\cventry
    {Statistics for Ecological Research} % Organization
    {} % Training Title 
    {} % Location
    {The Island School} % Date(s)
    {
      \begin{cvitems} % Description(s) of tasks/responsibilities
        \item {Three-part series covering introductory topics on common statics (descriptive and inferential) for \~30 high school juniors \& seniors}
      \end{cvitems}
    }


\end{cventries}


